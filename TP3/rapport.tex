\documentclass[12pt]{article}

\usepackage{amsfonts, amsmath, amssymb, amstext, latexsym}
\usepackage{graphicx, epsfig}
\usepackage[utf8]{inputenc}
\usepackage[french]{babel}
\usepackage{exscale}
\usepackage{amsbsy}
\usepackage{amsopn}
\usepackage{fancyhdr}
\usepackage{amsmath,amssymb}

\newcommand{\N}{\mathbb{N}}
\newcommand{\R}{\mathbb{R}}
\newcommand{\noi}{\noindent}
\newcommand{\dsp}{\displaystyle}
\newcommand{\iieme}{i^{\footnotesize \mbox{ème}}}
\newcommand{\jieme}{j^{\footnotesize \mbox{ème}}}
\newcommand{\jmunieme}{(j-1)^{\footnotesize \mbox{ème}}}
\newcommand{\dsum}[2]{\sum\limits_{#1}^{#2}}
\newcommand{\dprod}[2]{\prod\limits_{#1}^{#2}}

\def\ligne#1{\leaders\hrule height #1\linethickness \hfill}
% utilisation :
% \ligne{5}

\renewcommand{\theequation}{\thesection.\arabic{equation}}
\numberwithin{equation}{section}


\textheight 25cm
\textwidth 16cm
\oddsidemargin 0cm
\evensidemargin 0cm
\topmargin 0cm
\hoffset -0mm
\voffset -20mm


\pagestyle{plain}

\begin{document}

\baselineskip7mm

\noi ENSIMAG $2^{\footnotesize \mbox{ème}}$ année   \hfill Avril 2015


\vspace{1cm}


\begin{center}
{\Large \bf Traitement d'images\\ TP 3 \\ \texttt{Garcia Maxime, Palisse Benjamin, Petitbon Pierre}}
\end{center}

\vspace{5mm}

L'objectif de ce TP est de comparer deux méthodes de détection de contour d'une image, l'une par gradient l'autre par laplacien. Nous testerons les deux programmes sur des versions plus ou moins bruitées d'une image dont les contours intérieurs et extérieurs sont représentés sur les figures suivantes:


	\begin{figure}[!ht]
        \begin{center}
           \includegraphics[width=0.7]{formes1c1.pgm} 
           \caption{contour externe}
        \end{center}
    \end{figure}

	\begin{figure}[!ht]
        \begin{center}
           \includegraphics[width=0.7]{formes1c2.pgm} 
           \caption{contour interne}
        \end{center}
    \end{figure}

Nous testerons aussi nos programmes sur les images $tangram.pgm$ et $lacornou.pgm$, cette dernière étant bien plus compliqué.

 	\begin{figure}[!ht]
        \begin{center}
           \includegraphics[width=0.7]{formes1c1.pgm} 
           \caption{tangram.pgm}
        \end{center}
    \end{figure}

	\begin{figure}[!ht]
        \begin{center}
           \includegraphics[width=0.7]{formes1c2.pgm} 
           \caption{lacornou.pgm}
        \end{center}
    \end{figure}


\section{Détection par gradient}

\subsection{gradientNaif}

Cette détection de contour est obtenu en calculant le gradient de notre image, puis en considérant que les contours sont les points où le gradient est le plus important sans utiliser la suppression des non-maxima. En appliquant cette méthode sur l'image bruitée suivante (image simulant un bruit multiplicatif de type speckle), on obtient ceci:

	\begin{figure}[!ht]
        \begin{center}
           \includegraphics[width=0.7]{formes1sp.pgm} 
           \caption{image ayant un bruit multiplicatif de type speckle}
        \end{center}
    \end{figure}

	\begin{figure}[!ht]
        \begin{center}
           \includegraphics[width=0.7]{gradNaif_formes1bb5Seuil250.pgm} 
           \caption{image ayant un bruit multiplicatif de type speckle}
        \end{center}
    \end{figure}

	\begin{figure}[!ht]
        \begin{center}
           \includegraphics[width=0.7]{gradNaif_formes1bb5Seuil680.pgm} 
           \caption{image ayant un bruit multiplicatif de type speckle}
        \end{center}
    \end{figure}

On constate qu'en identifiant les points ayant un gradient supérieur à $250$ comme point de contour, le programme identifie bien la forme principale des différents objets de l'image mais il interprète l'intérieur des objets aussi comme des contours. En augmentant la valeur du seuil à 680, on constate que le seuil est trop grand car même les points qui sont réellement des contours tendent à disparaitre. En augmentant encore le seuil, c'est la totalité de la forme qui disparait. Cette méthode n'est donc pas bonne car le programme n'est pas capable de distinguer le contour de l'intérieur de l'objet.

\subsection{gradient Prewitt}
Cette fois-ci, on supprime les non maxima et on fait un seuillage par hystérésis en utilisant le gradient de Prewitt.

\section{Détection par Laplacien}

On utilise la détection de contour sur l'image $formes1bb25.pgm$. Si l'on prend $\sigma_1=10$ et $\sigma_2=6.25$(le rapport des $2$ faisant bien $1.6$), on obtient

	\begin{figure}[!ht]
        \begin{center}
           \includegraphics[width=0.7]{laplacien_formes1bb25-10-625.pgm} 
           \caption{laplacien avec 10 et 6.25 pour valeur de filtre}
        \end{center}
    \end{figure}

Les formes principales des contours ressortent, cependant le programme rajoute de nombreux contours "parasites" autour des formes à cause du bruit. Les contours reconnus par le programme sont plus arrondis que ne le sont les originaux réellement.

Avec comme paramètre des filtres lors de la différence de gaussienne $\sigma_1=30$ et $\sigma_2=18.75$, on obtient les contours suivants:

	\begin{figure}[!ht]
        \begin{center}
           \includegraphics[width=0.7]{laplacien_formes1bb25-30-1875.pgm} 
           \caption{laplacien avec 30 et 18.75 pour valeur de filtre}
        \end{center}
    \end{figure}

Les contours sont très apparents mais plus arrondis que l'image originale, cela est notamment très visible sur le rectangle en bas. Les traits un peu "tremblants" de la figure au milieu en haut dans l'image initiale apparaissent ici très courbés. La figure de droite est devenu un rond. Par rapport au cas précédent, les contours sont plus arrondis, et donc moins fidèle à la réalité, mais les contours superflues des bords ont disparu. 

Pour l'image tangram, avec $\sigma_1=5$ et $\sigma_2=3.125$:

	\begin{figure}[!ht]
        \begin{center}
           \includegraphics[width=0.7]{laplacien_tangram-5-3125.pgm} 
           \caption{laplacien avec 5 et 3.125 pour valeur de filtre}
        \end{center}
    \end{figure}

Les triangles initiaux sont reconnus par le programme, ainsi que le petit effet de relief qu'il y a en chacun d'eux. Cependant, de nombreux contours d'objets qui n'étaient pas dans l'image initiale apparaissent, comme dans les cas précédents. Le coté gauche à l'origine vide est alors remplie par des objets.

Si l'on augmente les valeurs des filtres, $\sigma_1=12$ et $\sigma_2=7.5$, on obtient:

	\begin{figure}[!ht]
        \begin{center}
           \includegraphics[width=0.7]{laplacien_tangram-12-7-5.pgm} 
           \caption{laplacien avec 12 et 7.5 pour valeur de filtre}
        \end{center}
    \end{figure}

Les triangles de la figure initiale apparaissent aussi dans le résultat produit par notre algorithme. Cependant, ils prennent une forme plus ronde sur les bords, et l'effet de relief disparait. Les contours apparaissant sur les bords du aux bruits sont moins nombreux.

Pour l'image lacornou, en prenant $\sigma_1=8$ et $\sigma_2=5$

	\begin{figure}[!ht]
        \begin{center}
           \includegraphics[width=0.7]{laplacien_lacornou-8-5.pgm} 
           \caption{laplacien avec 8 et 5 pour valeur de filtre}
        \end{center}
    \end{figure}

On constate que pour une image avec ce niveau de détail, notre programme ne fournit pas des contours satisfaisant.
\end{document}
